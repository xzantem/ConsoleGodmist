% *************** Zakończenie ***************

%***************************************************************************************
% W tym miejscu znajdują się polecenia odpowiedzialne za tworzenie
% spisu ilustracji, spisu treści oraz streszczenia pracy
%***************************************************************************************

%spis rysunków
\addcontentsline{toc}{chapter}{Spis rysunków}
\listoffigures
\newpage

%spis tablic
\addcontentsline{toc}{chapter}{Spis tablic}
\listoftables
\newpage

% %streszczenie
% \addcontentsline{toc}{chapter}{Streszczenie}
% \noindent
% {\footnotesize{}\textbf{Wyższa Szkoła Informatyki i Zarządzania z siedzibą w Rzeszowie\\
% Kolegium Informatyki Stosowanej}
% \vspace{30pt}

% \begin{center}
% \textbf{Streszczenie pracy dyplomowej inżynierskiej}\\
% \temat
% \end{center}

% \vspace{30pt}
% \noindent
% \textbf{Autor: \autor
% \\Promotor: \promotor
% \\Słowa kluczowe: tutaj umieść słowa kluczowe}
% \vspace{40pt}
% \\Treść streszczenia, czyli kilka zdań dotyczących treści pracy dyplomowej w języku polskim.
% \vspace{80pt}

% \noindent
% \textbf{The University of Information Technology and Management in Rzeszow\\
% Faculty of Applied Information Technology}
% \vspace{30pt}

% \begin{center}
% \textbf{Thesis Summary\\}
% Tytuł pracy w języku angielskim
% \end{center}

% \vspace{30pt}
% \noindent
% \textbf{Author: \autor
% \\Supervisor: \promotor
% \\Key words: tutaj umieść słowa kluczowe}
% \vspace{40pt}
% \\Treść streszczenia, czyli kilka zdań dotyczących treści pracy dyplomowej w języku angielskim - tłumaczenie tekstu z języka polskiego.
% }

% *************** Koniec pliku back.tex ***************
