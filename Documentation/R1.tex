% ********** Rozdział 1 **********
\chapter{Opis założeń projektu}
\section{Cele projetu}
%\subsection{Tytuł pierwszego podpunktu}

Projekt będzie dotyczył stworzenia gry komputerowej akcji z walką turową i elementami RPG, której głównym celem będzie dostarczenie użytkownikowi emocjonującej rozgrywki w walkach, eksploracji lochów, wykonywaniu zadań i walka z bossami. 
Problemem rozwiązywanym przez projekt jest deficyt gier komputerowych o tematyce roguelike, a także brak gier oferujących przystępny, ale angażujący system rozwoju postaci, który nie jest zbyt skomplikowany ani zbyt prosty. 
Wiele osób poszukuje gier, które nie wymagają skomplikowanego uczenia się, ale oferują angażującą rozgrywkę i możliwość rozwoju postaci w zależności od wybranej strategii. 
Prosty, ale wciągający system walki sprawi, że gra będzie dostępna dla graczy o różnych doświadczeniach z grami komputerowymi, co zwiększa jej potencjalną popularność. 
Analiza rynku gier pokazuje, że gracze cenią sobie gry z przystępnym poziomem trudności, po które można łatwo sięgnąć i równie łatwo odłożyć na później. 
Aby rozwiązać problem, wymagana jest wiedza na temat rynku gier, preferencji graczy, projektowania i programowania gier oraz testowania ich. 
W celu rozwiązania projektu należy: zaplanować i przygotować koncepcję gry; zaprojektować system walki i inne mechaniki; zaprojektować inne elementy takie jak przedmioty, umiejętności, lub przeciwnicy; zaimplementować mechaniki, funkcjonalności oraz 
interfejs użytkownika; przetestować oraz zoptymalizować grę; wydać finalną wersję gry. Wynikiem pracy będzie gotowy program komputerowy w postaci gry.

\section{Wymagania funkcjonale i niefunkcjonalne}
 
\noindent \textbf{Wymagania funkcjonalne:}

\noindent Na liście wymagań funkcjonalnych znajdują się,
\begin{itemize}
    \item Tworzenie postaci.
    \item System statystyk, poziomów i umiejętności.
    \item Efekty pasywne, efekty statusu
    \item System walki.
    \item System eksploracji lochów: poruszanie się, pola z walką, ogniska, skrzynie, rośliny i pułapki
    \item Kupowanie, używanie, tworzenie i inne operacje na przedmiotach i ekwipunku.
    \item System ekwipunku i powiązanych mechanik (ulepszanie, przekuwanie, wstawianie ulepszeń w postaci galdurytów).
    \item System innych przedmiotów użytkowych: mikstur, galdurytów, torb z przedmiotami i innych.
    \item Generowanie przeciwników.
    \item System postaci niezależnych (NPC) i ich usług
    \item System misji
    \item Obsługa plików JSON jako baza danych NoSQL.
    \item Zapis stanu gry.
    \item Interfejs użytkownika (Spectre.Console)
\end{itemize}

\noindent \textbf{Wymagania niefunkcjonalne:}

\noindent Na liście wymagań niefunkcjonalnych znajdują się,
\begin{itemize}
    \item Wydajność.
    \item Responsywność.
    \item Zgodność z platformą Windows.
    \item Dostępność.
    \item Optymalizacja pamięci.
    \item Możliwość rozbudowy.
    \item Użyteczność.
    \item Niezawodność.
    \item Przenośność.
\end{itemize}


% ********** Koniec rozdziału **********
