% ********** Rozdział 5 **********
\chapter{Podsumowanie}

Wynikiem projektu jest wysoce rozwinięta, ale prosta do obsługi gra konsolowa RPG. Prostość w obsłudze i złożoność systemów, wraz z
proceduralnością wbudowaną w mechaniki gry zapewni godziny rozgrywki i zwiększy popularność gry. System zapisów gry pozwoli na tworzenie ciągłych 
rozgrywek, gdzie gracze angażują się w wiele poziomów i rowzijają jedną postać bardzo długo, co skutkuje satysfakcją i chęcią do dalszej gry.

\section{Zrealizowane prace}

W ramach projektu zrealizowano następujące prace:

\begin{itemize}
    \item \textbf{Opracowanie i implementacja głównych klas:} Stworzono klasy, takie jak \texttt{PlayerCharacter}, \texttt{EnemyCharacter}, \texttt{Battle}, oraz \texttt{BattleManager}, które stanowią fundament mechaniki gry.
    \item \textbf{Implementacja systemu NPC:} Stworzono klasy reprezentujące postacie niezależne (NPC), takie jak \texttt{Enchanter}, \texttt{Blacksmith} oraz \texttt{Alchemist}. Każda z tych klas posiada unikalne metody interakcji z graczem oraz system lojalności.
    \item \textbf{Zarządzanie ekwipunkiem:} Opracowano system zarządzania ekwipunkiem gracza oraz NPC, umożliwiający kupno, sprzedaż i tworzenie przedmiotów.
    \item \textbf{System walki:} Zaimplementowano mechanikę walki, w tym klasy \texttt{Battle}, \texttt{BattleManager} oraz \texttt{BattleUser}, które zarządzają przebiegiem walki oraz interakcjami między postaciami.
    \item \textbf{Zarządzanie danymi:} Opracowano system zapisywania i ładowania stanu gry, co pozwala graczom na kontynuowanie rozgrywki w późniejszym czasie.
    \item \textbf{Wprowadzenie mechaniki rozwoju postaci:} Wprowadzono system rozwoju postaci, w tym zdobywania doświadczenia, honoru oraz modyfikatorów, które wpływają na interakcje w grze.
    \item \textbf{Implementacja systemu zadań:} Opracowano system zadań, które dodają dodatkowy cel rozgrywce i dodatkowe nagrody dla graczy.
    \item \textbf{Eksploracja lochów:} Wprowadzono system eksploracji proceduralnie generowanych lochów, które zmniejszają powtarzalność rozgrywki.

\end{itemize}

\section{Planowane prace rozwojowe}

W przyszłości planowane są następujące prace rozwojowe:

\begin{itemize}
    \item \textbf{Rozbudowa systemu questów:} Dodanie nowych zadań oraz mechanik związanych z ich realizacją, co zwiększy interaktywność i zaangażowanie graczy. Wprowadzenie angażującej i oryginalnej fabuły głównej.
    \item \textbf{Optymalizacja algorytmów walki:} Udoskonalenie sztucznej inteligencji przeciwników oraz wprowadzenie nowych umiejętności i strategii walki, takich jak: umiejętności pasywne, fazy bossów, walka drużynowa.
    \item \textbf{Optymalizacja UX/UI:} Dodanie większej ilości informacji zwrotnej dla gracza przy interakcjach z interfejsem, Wprowadzenie bardziej czytelnego interfejsu.
    \item \textbf{System osiągnięć:} Implementacja systemu osiągnięć, który nagradza graczy za różne osiągnięcia w grze, co zwiększy motywację do eksploracji i interakcji z grą.
    \item \textbf{Drzewka umiejętności:} Implementacja drzewek umiejętności, które pozwolą graczom na wybranie ścieżki rozwoju i zwiększą kompleksowość rozwoju postaci.
\end{itemize}



% ********** Koniec rozdziału **********

